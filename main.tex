\documentclass{report}
\usepackage[a4paper, left=3cm, right=3cm, top=2cm, bottom=2cm]{geometry}
\usepackage{graphicx} % Required for inserting images
\usepackage{float}
\usepackage{amsmath}
\usepackage{hyperref}

\usepackage{amsmath}
\newcommand{\Mod}[1]{\ (\mathrm{mod}\ #1)}

\title{Stochastik WS23}
\author{Meiers Thierry}
\date{November 2023}

\begin{document}

\maketitle

\chapter*{Grundlegendes} 

\subsection*{Kombinatorische Formeln}

\subsubsection{Ohne Wiederholungen}
$x = (x_1, ..., x_n)$ Vektor mit verschiedenen Einträgen.
\begin{enumerate}
    \item Es gibt $n!:=n\cdot(n-1)\cdot...\cdot1$ verschiedene Vektoren der Länge n, die dieselben Einträge wie x haben.
    \item Zieht man k-mal ohne Zurücklegen, so gibt es genau $\frac{n!}{(n-k)!}$ mögliche Ausgänge, wenn man die Reihenfolge der gezogenen Einträge beachtet.
    \item Betrachtet man die Reihenfolge nicht, so gibt es genau $\binom{n}{k}=\frac{n!}{k!\cdot(n-k)!}$ mögliche Ausgänge.
\end{enumerate}

\subsubsection{Mit Wiederholungen}
$x = (x_1, ..., x_n)$ Vektor.
\begin{enumerate}
    \item Sind genau r Einträge verschieden mit Häufigkeit $k_1, ..., k_r$, so gibt es genau $\frac{n!}{k_1!\cdot...\cdot k_r}$ verschiedene Möglichkeiten, die Einträge in einer Reihenfolge zu bringen.
    \item Sind alle Einträge verschieden, und zieht man k-mal mit Zurücklegen, so gibt es genau $n^k$ mögliche Ausgänge, wenn man die Reihenfolge betrachtet. 
    \item Betrachtet man die Reihenfolge nicht, so gibt es genau $\binom{n+k-1}{k}=\frac{(n+k-1)!}{k!\cdot(n-1)!}$ mögliche Ausgänge.
\end{enumerate}

\subsubsection*{Stirling-Formel}
Wie groß ist eigentlich n! ?
\[\displaystyle \lim_{n \to \infty} \frac{n!}{\binom{n}{e}^n \sqrt{2\pi n}} = 1 \Rightarrow n! \approx \binom{n}{e}^n \sqrt{2\pi n}\]
Bew:
\[log(n!) = \sum_{i=1}^n log(i) = \int_{1}^{n} log(x)dx + O(log(n)) = x log(x) - x + O(log(n)) = n log(n) - n + O(log(n)) \]
\[n! = e^{log(n!)} = e^{nlog(n)-n}O(n) = O(n)\binom{n}{e}^n\]

\section*{Wahrscheinlichkeitsräume und Zufalls-Variablen}

\subsection*{Zufallsvariablen}
Sei E eine Menge. Ein zufälliges Element X von E heißt (E-Wertige) Zufalls-variable. Hierbei bedeutet zufällig, dass man X eine Verteilung zuordnen kann.

\subsubsection{Verteilung einer Zufallsvariable}
\begin{itemize}
    \item Falls E ist höchstens abzählbar, so heißt X diskret und $x \rightarrow P(X=x) $ \textbf{Zähldichte} der Verteilung von X.
    \item Falls $E \subseteq R$ und für $a < b$ $P(X \in (a, b)) = \int_a^b f(x)dx$ so heißt X \textbf{stetig} und f \textbf{Dichte} der (stetigen) Verteilung von X.
\end{itemize}

\subsubsection*{Wahrscheinlichkeitsverteilung}
Sei $P(E)$ die Potenzmenge von $E, P : A \subset P(E) \rightarrow [0, 1]$ eine Abbildung. 
\begin{itemize}
    \item $P(E) = 1$
    \item $\forall a \in A \rightarrow P(a^c) = 1 - P(a)$
    \item Für paarweise disjunkte $A_1, ..., A_n$ ist $P(A_1  \bigvee ... \bigvee A_n) = \sum_{n=1}^\infty p(An)$
\end{itemize}

\subsection*{Einschluss-Ausschluss-Formel}
\url{https://www.aleph1.info/?call=Puc&permalink=ema22_4_2_Z6}

\subsection*{Bild-Verteilungen}
Ist X eine E-wertige ZV und $h : E → E'$, so ist $Y := h(X)$ eine $E'$-wertige ZV und
 \[P(Y \in B) = P(X \in h^{-1}(B))\]
Die Verteilung von Y ist Bildverteilung der Verteilung von X unter h

\section*{Unabhängigkeit}
Seien $X_1, ..., X_n$ Zufallsvariablen. Falls $P(X_1 \in A_1, ..., X_v \in A_n) = P(X_1 \in A_1) \cdot ... \cdot P(X_n \in A_n)$ für alle $A_1, ..., A_n$, so heißen $X_1, ..., X_n$ unabhängig.

\chapter*{Verteilungen und deren Eigenschaften}

\section*{Zufallszahlen}
Sei $E$ endlich und $e_1 \in E$ ein Startzustand. Des weiterem sei $T : E \rightarrow W$ eine Übergangsfunktion. Dann heißt $(x_1, x_2, ...)$ mit $x_1 := e_1 \in E$ und $x_{n+1} = T(x_n), n = 1, 2, ...$ eine Folge von Pseudozufallszahlen und $(T, e_1)$ ein Pseudo-Zufallszahlengenerator. 
\subsection*{Kongruenzgenerator}
$e_1 \in E = N_0$ Gib es $a, c, m \in N$ so dass $T(x) = ax + c \mod m$ so heißt $(T, e_1)$ linearer Kongruenzgenerator. Ist $c = 0$ so heißt $(T, e_1)$ multiplikativer Kongruenzgenerator.
\subsection*{Satz von Euler}
Ist $\delta(m)$ die Anzahl der zu m teilfremden Zahlen in $\{1, ..., m - 1\}$ und sind a, m, teilerfremd, so gilt \[a^{\delta(m)} = 1 \mod m\]

\section*{Grundlegendes}
\subsection*{Verteilungsfunktion}
Sei X eine $E \subseteq R$-Wertige Zufallsvariable. Dann heißt \[x \rightarrow F_x(X) := P(X \leq x)\] Verteilungsfunktion von X. Sei X reelwertige Zufallsvariable, dann gilt

\section*{Uniforme Verteilungen, Laplace-Experimente}
\section*{Die Binomialverteilung}
\section*{Die Normalverteilung}
\section*{Die Exponentialverteilung}
\section*{Pseudo-Zufallszahlen mit beliebiger Verteilung}

\end{document}
